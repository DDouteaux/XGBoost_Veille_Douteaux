%%---------------------------------------------------------------%%
%                                                                 %
%   Fichier de toutes les configurations basiques utilisateurs    %
%                                                                 %
%%---------------------------------------------------------------%%

%
% 1 - Positionnement des TOC - LOF - LOT dans le document
%
\newcommand\whereTOC{beginning}          % Parmis beginning - end
\newcommand\whereLOF{end}                % Parmis beginning - end
\newcommand\whereLOT{end}                % Parmis beginning - end
\newcommand\TOCLOFTNumStyle{Roman}       % Parmis Roman - roman - Arabic - arabic
\newcommand\LOFLOTSeparator{\vfill}      % Séparateur après une LOF ou LOT
\newcommand\TOCSeparator{\newpage}       % Séparateur après un TOC
\newcommand\inclureTOC{true}             % Mettre la TOC (true) ou pas (false)
\newcommand\inclureLOF{false}            % Mettre la LOF (true) ou pas (false)
\newcommand\inclureLOT{false}            % Mettre la LOT (true) ou pas (false)

%
% 2 - Fichiers à inclure
%
\newcommand\inclureIntroduction{false}   % Mettre l'introduction (true) ou pas (false)    -> Position imposée
\newcommand\inclureConclusion{false}     % Mettre la conclusion (true) ou pas (false)     -> Position imposée
\newcommand\inclureRemerciements{false}  % Mettre les remerciements (true) ou pas (false)
\newcommand\inclureResume{false}         % Mettre le résumé (true) ou pas (false)
\newcommand\inclureAbstract{false}       % Mettre le résumé anglais (true) ou pas (false)
\newcommand\inclureReferences{false}     % Mettre les références (true) ou pas (false)
\newcommand\whereResume{beginning}       % Parmis beginning - end
\newcommand\whereRemerciements{beginning}% Parmis beginning - end
\newcommand\whereReferences{end}         % Parmis beginning - end

%
% 3 - Informations pour en-têtes et pieds de page
%
\newcommand\footerRight{\textbf{Année 2017}   }                    % À gauche séparation du footer
\newcommand\footerLeft{\textbf{\thepage}}                          % À droite séparation du footer
\newcommand\headerText{\textbf{Principe et applications de XGBoost}}   % Texte de l'en-tête
\newcommand\headerLogo{logo}                                       % Logo de l'en-tête (optionnel)

%
% 4 - Page de titre à charger
%
\newcommand\titlePage{picture}                                     % Voir les noms dans conf/Pages_titre/

%
% 5 - Le type de sections désiré
%

% 5.1 - Aspect d'une partie
\newcommand\partFormat{page}
\newcommand\partModele{default}
% 5.2 - Aspect d'une section
\newcommand\sectionFormat{greatNumber}
\newcommand\sectionNumber{arabic}
\newcommand\sectionNumberStyle{textbfgray}
\newcommand\sectionSeparator{}
\newcommand\sectionSeparatorStyle{thColor}
\newcommand\sectionTextStyle{textbfc}
% 5.3 - Aspect d'une sous-section
\newcommand\subsectionFormat{light}
\newcommand\subsectionNumber{arabic}
\newcommand\subsectionNumberStyle{textbfc}
\newcommand\subsectionSeparator{~~$\bullet$~~}
\newcommand\subsectionSeparatorStyle{thColor}
\newcommand\subsectionTextStyle{thColor}
% 5.4 - Aspect d'une sous-sous-section
\newcommand\subsubsectionFormat{default}
\newcommand\subsubsectionNumber{arabic}
\newcommand\subsubsectionNumberStyle{thColor}
\newcommand\subsubsectionSeparator{~~}
\newcommand\subsubsectionSeparatorStyle{textbf}
\newcommand\subsubsectionTextStyle{textit}
% 5.5 - Aspect d'un paragraphe
\newcommand\paragrapheFormat{default}
\newcommand\paragrapheNumber{none}        % Pas de numérotation (none), autrement {Roman ; roman ; arabic ; Arabic}
\newcommand\paragrapheNumberStyle{}
\newcommand\paragrapheSeparator{}
\newcommand\paragrapheSeparatorStyle{}
\newcommand\paragrapheTextStyle{}
% 5.6 - Aspect d'un sous-paragraphe
\newcommand\subparagrapheFormat{default}
\newcommand\subparagrapheNumber{none}     % Pas de numérotation (none), autrement {Roman ; roman ; arabic ; Arabic}
\newcommand\subparagrapheNumberStyle{}
\newcommand\subparagrapheSeparator{}
\newcommand\subparagrapheSeparatorStyle{}
\newcommand\subparagrapheTextStyle{textic}
% 5.7 - Aspect de la page d'annexe
\newcommand\annexeModele{default}         % Aucune (none) ou voir dans conf/Pages_annexes/

%
% 6 - Pour les sections avec images, configuration des images
%
\newcommand\imageSectionI{default1}
\newcommand\imageSectionII{default2}
\newcommand\imageSectionIII{default3}
\newcommand\imageSectionIV{default4}
\newcommand\imageSectionV{default5}