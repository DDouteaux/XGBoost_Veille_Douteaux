\subsection{Perte et complexité}
Avant d'aborder la question des GBM, nous allons nous intéresser à deux grands concepts importants en machine learning et qui sont tout particulièrement gardés à l'\oe il dans le cas de XGBoost et du boosting de manière générale.

Nous avons vu que l'idée du Boosting est d'adapter l'ensemble d'apprentissage à chaque étape pour chercher des classifieurs qui vont se compléter, ie. moyenner leurs erreurs. Il s'agit en fait de \og coller aux données\fg{} du mieux que possible (mais sans sur-apprendre!). Nous verrons que cet aspect est lié à la notion de perte.

De même, nous avons vu que l'idée n'est pas d'apprendre des modèles intermédiaires très puissants, mais que l'ensemble soit performant. Ainsi, nous verrons que le fait de prendre des modèles simples est lié à la notion de complexité.

\subsection{Fonction de perte}
Avant d'aller plus loin, regardons formellement quelle est l'idée théorique derrière le Boosting. Ce dernier cherche en fait à optimiser (minimiser) une fonction objectif. Cette fonction peut s'écrire en deux termes :
\begin{equation}
\textnormal{objectif}(\Theta)=\mathcal{L}(\Theta)+\Omega(\Theta)}
\label{eqn:obj}
\end{equation}
La relation~(\ref{eqn:obj}) fait apparaître deux termes \mysidenote{Dans la relation~(\ref{eqn:obj}), la variable $\Theta$ représente juste le modèle pour lequel on fait les calculs.}:\begin{itemize}
	\itemperso{$\mathcal{L}(\Theta)$}La fonction de perte
	\itemperso{$\Omega(\Theta)$}La fonction de complexité
\end{itemize}
Ce sont ces deux fonctions (et leur intérêt!) qui vons nous intéresser par la suite. Pour cela, nous allons considérer la situation cobaye de la Figure~\ref{fig:cobaye}.

\begin{figure}[h]
	\begin{margincap}
	  \centering
	  \begin{tikzpicture}[x=0.4cm,y=0.4cm,]
\tikzset{cross/.style={cross out, draw=black, minimum size=2*(#1-\pgflinewidth), inner sep=0pt, outer sep=0pt},
%default radius will be 1pt. 
cross/.default={5pt}}

\draw (1,2) node[cross,red] {};
\draw (2,2.5) node[cross,red] {};
\draw (2.8,2.2) node[cross,red] {};
\draw (3.5,3) node[cross,red] {};
\draw (3.8,5) node[cross,red] {};
\draw (4.2,7) node[cross,red] {};
\draw (5,6.8) node[cross,red] {};
\draw (6,6.5) node[cross,red] {};
\draw (6.8,6.7) node[cross,red] {};
\draw (8,7) node[cross,red] {};
\end{tikzpicture}
	  \caption{Situation de test pour comprendre les notions de perte et complexité (d'après~\cite{bib:xgboost-main})}
	  \label{fig:cobaye}
	\end{margincap}
\end{figure}

Sur la Figure~\ref{fig:cobaye}, nous avons également représenté en \raisebox{4pt}{\tikz{\draw[dashed, gray](0,0) -- (.75,0);}} un modèle qui représenterait un bon compromis.

\subsubsection{Notion de perte}

Cette fonction (notée $\mathcal{L}$ dans la relation~(\ref{eqn:obj})), représente la qualité prédictive du modèle sur l'ensemble d'apprentissage. Attention, il est important de voir qu'on s'intéresse ici à l'adéquation à l'ensemble d'apprentissage, mais pas sur l'ensemble de test (puisqu'on ne le connaît pas à priori). Les fonctions communément utilisées sont par exemple l'erreur MSE (liée à la norme $\mathcal{L}_2$) ou une expression de perte logistique.

\begin{figure}[h]
	\begin{margincap}
	  \centering
	  \subsection{Fonction de coût personnalisée}
\label{sec:cout-perso}
Nous avons à plusieurs reprise mentionné la possibilité d'utiliser les fonctions de pertes personnalisées avec XGBoost. Un exemple est fournit ci-dessous où l'on définit manuellement la fonction de log-vraissemblance pour la perte\mysidenote{La mesure de log-vraissemblance fait en réalité partie des options possibles par défaut dans XGBoost, elle est cependant présentée sous cette forme \og manuelle\fg{} ici à titre d'exemple. Pour mémoire, la manière de l'appeler serait de faire \texttt{objective = "binary:logistic".}}.

Cet exemple est réalisé en R, mais le fonctionnement sera le même pour d'autres langages.
\begin{lstlisting}[language=R]
loglossobj <- function(preds, dtrain) {
  # On extrait les labels de l'ensemble d'apprentissage
  labels <- getinfo(dtrain, "label")
  # Calcul du gradient et de la partie de la hessienne utiles 
  # dans les relations de boosting.
  preds <- 1/(1 + exp(-preds))
  grad <- preds - labels
  hess <- preds * (1 - preds)
  # Renvoie des résultats sous forme de liste.
  return(list(grad = grad, hess = hess))
}

# Entrainement du modèle avec notre méthode
model <- xgboost(data = train$data, label = train$label, nrounds = 2, objective = loglossobj, eval_metric = "error")
\end{lstlisting}

On remarque donc que la définition est des plus simples\mysidenote{Tant que l'on peut trouver des expressions algébriques...}, dans la mesure où les seuls éléments demandés sont de pouvoir exprimer le gradient et la hessienne de la fonction de perte.

	  \caption{Une situation où la perte n'est pas optimisée (d'après~\cite{bib:xgboost-main})}
	  \label{fig:loss}
	\end{margincap}
\end{figure}

La Figure~\ref{fig:loss} propose une situation où la perte induite pas le modèle est plus importante que ce qu'il est possible de dans le cas optimal. L'idée de cette fonction $\mathcal{L}$ est donc de pouvoir détecter ces situations et d'estimer si un nouveau modèle améliore ou non cette situation.

\subsubsection{Notion de complexité}

Cette fonction (notée $\Omega$ dans la relation~(\ref{eqn:obj})), est aussi appelée terme de régularisation. Elle permet de représenter la complexité du modèle. L'idée est ici qu'un modèle trop précis (\textit{overfitting}) sera généralement trop complexe. Ainsi, ce terme permettra de contrôler la complexité du modèle pour éviter les phénomènes d'\textit{overfitting} sur les données d'apprentissage et éviter une chute importante de la qualité prédictive du modèle sur l'ensemble d'apprentissage.

Un exemple de fonction de complexité vous sera proposé lors de la Section~\ref{sec:gradient-boosting}.

\begin{figure}[h]
	\begin{margincap}
	  \centering
	  \begin{tikzpicture}[x=.5cm,y=.5cm,]
\tikzset{cross/.style={thick, cross out, draw=bluenight, minimum size=2*(#1-\pgflinewidth), inner sep=0pt, outer sep=0pt},
%default radius will be 1pt. 
cross/.default={2.5pt}}

\draw [thick, themeColor] (-.8,1.9) -- (1.5,1.9) -- (1.5,2.5) -- (3.5,2.5) -- (3.5,4.2) -- (4.2,4.2) -- (4.2,5.7) --  (5.2,5.7) -- (5.2,5.4) --   (6.2,5.4) -- (6.2,5.7) -- (8,5.7);
\draw [dashed, gray] (-.8,2.1) -- (3.8,2.1) -- (3.8,5.7) -- (8.2,5.7);
\draw[thick,-{Latex[length=2mm]}] (-.8,0) -- (9.5,0);
\draw[thick,-{Latex[length=2mm]}] (-.8,0) -- (-.8,7);
\node [left] at (-.8,7) {Intérêt utilisateur\hspace*{.2cm}};
\node [right] at (9.5,0) {\hspace*{.2cm}Temps};
\draw (.2,1.9) node[cross] {};
\draw (1,2) node[cross] {};
\draw (2,2.5) node[cross] {};
\draw (2.8,2.2) node[cross] {};
\draw (3.5,3) node[cross] {};
\draw (3.8,4) node[cross] {};
\draw (4.2,5) node[cross] {};
\draw (5,5.6) node[cross] {};
\draw (6,5.5) node[cross] {};
\draw (6.8,5.7) node[cross] {};
\draw (7.8,5.8) node[cross] {};
\end{tikzpicture}
	  \caption{Une situation où la complexité n'est pas optimisée (d'après~\cite{bib:xgboost-main})}
	  \label{fig:complexity}
	\end{margincap}
\end{figure}

Le modèle qui était représenté en gris sur les Figures~\ref{fig:cobaye} à~\ref{fig:complexity} correspondait en fait à un bon compromis entre $\mathcal{L}$ et $\Omage$.
