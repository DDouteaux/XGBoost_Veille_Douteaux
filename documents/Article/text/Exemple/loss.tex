\subsection{Fonction de coût personnalisée}
Nous avons à plusieurs reprise mentionné la possibilité d'utiliser les fonctions de pertes personnalisées avec XGBoost. Un exemple est fournit ci-dessous où l'on définit manuellement la fonction de log-vraissemblance pour la perte\mysidenote{La mesure de log-vraissemblance fait en réalité partie des options possibles par défaut dans XGBoost, elle est cependant présentée sous cette forme \og manuelle\fg{} ici à titre d'exemple. Pour mémoire, la manière de l'appeler serait de faire \texttt{objective = "binary:logistic".}}.

Cet exemple est réalisé en R, mais le fonctionnement sera le même pour d'autres langages.
\begin{lstlisting}[language=R]
loglossobj <- function(preds, dtrain) {
  # On extrait les labels de l'ensemble d'apprentissage
  labels <- getinfo(dtrain, "label")
  # Calcul du gradient et de la partie de la hessienne utiles 
  # dans les relations de boosting.
  preds <- 1/(1 + exp(-preds))
  grad <- preds - labels
  hess <- preds * (1 - preds)
  # Renvoie des résultats sous forme de liste.
  return(list(grad = grad, hess = hess))
}

# Entrainement du modèle avec notre méthode
model <- xgboost(data = train$data, label = train$label, nrounds = 2, objective = loglossobj, eval_metric = "error")
\end{lstlisting}

On remarque donc que la définition est des plus simples\mysidenote{Tant que l'on peut trouver des expressions algébriques...}, dans la mesure où les seuls éléments demandés sont de pouvoir exprimer le gradient et la hessienne de la fonction de perte.
