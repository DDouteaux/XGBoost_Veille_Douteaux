L'objectif de cet article est de présenter XGBoost. Mais qu'est-ce que XGBoost? Il s'agit d'une méthode de machine learning apparue il y a trois ans et dérivant des méthodes dites de boosting. Cette méthode est présentée par ses créateurs comme étant :
\begin{itemize}
	\item[\thColor{\faFlag~~Flexible}]Prise en compte de plusieurs thématiques de machine learning.
	\item[\thColor{\faCube~~Portable}]Utilisable sous toutes les plateformes (Windows, Linux, MAC)
	\item[\thColor{\faCommentsO~~Multi-langages}]L'algorithme a été porté en Python, JAVA (Spark), C++,...
	\item[\thColor{\faCloud~~Distribuée}]Depuis deux ans, l'algorithme est utilisable avec Hadoop et Spark.
	\item[\thColor{\faRocket~~Perfomante}]Cet algorithme est donné pour être plus rapide que les algorithmes de sa famille et fournir de même meilleurs résultats.
\end{itemize}
Mais avant toute chose, que veut exactement dire l'acronyme \og XGBoost\fg?
\begin{center}
E\textbf{\textcolor{bluenight}{X}}treme \textbf{\textcolor{bluenight}{G}}radiant \textbf{\textcolor{bluenight}{Boost}}ing
\end{center}
Ainsi, la méthode XGBoost s'organise autours de trois points essentiels :
\begin{itemize}
	\itemperso{Boosting}Il s'agit d'une famille d'algorithme utilisés initiallement en apprentissage supervisé. Le principe du boosting sera détaillé en Section~\ref{sec:boosting}.
	\itemperso{Gradient Boosting}Il s'agit d'une version du boosting dans laquelle l'objectif sera d'optimiser une fonctione faisant apparaître des gradients. Les détails de cette idée seront détaillées en Section~\ref{sec:gradient-boosting}.
	\itemperso{\og Extreme\fg}Ce qualificatif signifie que la recherche de performances est poussée au maximum pour cette méthode, comme nous l'étudierons en Section~\ref{sec:perf}.
\end{itemize}
Ainsi, nous allons dans un premier temps présenter les grandes lignes théoriques derrière cet algorithme avant de partir dans l'étude des implémentations et des applications de XGBoost.

Nous nous intéresserons également aux aspects plus techniques de XGBoost, en regardant en détails les principaux paramètres de cette méthode d'apprentissage.

Il sera également intéressant de regarder quels sont les usages réels de cette méthode, afin de nous forger un avis sur sa pérennité.

\vskip 1cm
\textit{Tous les supports présentés en séance ainsi que le code du site web associé à ce rendu peuvent être trouvés sur ma page Github : \url{https://github.com/DDouteaux/XGBoost_Veille_Douteaux/}. N'hésitez pas non plus à retrouver le compte Twitter de cette veille, @DDouteaux.}
