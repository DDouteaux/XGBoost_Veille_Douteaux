\subsection{Valeurs manquantes}
Les valeurs manquantes sont un problème courant en analyse de données. La plupart des algorithmes proposent aujourd'hui deux solutions, soit l'utilisateur pré-traite ses données avant de les utiliser et retire ou modifie au besoin les données manquantes, soit il \og laisse faire\fg{} l'algorithme, au risque que la solution ne lui convienne pas.

XGBoost se démarque ici en proposant un traitement non binaire (ie. on n'assigne pas par défaut \texttt{null} ou \texttt{0},...) pour traiter les données manquantes. À la place, l'algorithme propose d'assigner une direction aux valeurs manquantes plutôt qu'une valeur numérique particulière.

Ceci signifie que lorsqu'une découpe est réalisée selon une variable proposant des valeurs manquantes, l'algorithme va classer les données manquantes \og à gauche ou à droite\fg{} dans les fils du n\oe uds. A posteriori de ce classement, XGBoost va regarder quel classement fournirait le meilleur gain et re-répartir ces instances de manière à maximiser ce dernier\mysidenote{Pour les personnes intéressées, l'algorithme explicite est précisé dans l'article de T. \textsc{Chen} (Algorithme 3) \cite{bib:xgboost-article}}.

Ce comportement est activé par défaut pour XGBoost, et a pour intérêt que l'algorithme va apprendre une direction optimale au lieu de simplement considérer toutes les valeurs inconnues comme identiques ou alors les rejeter.
\subsection{Cross-validation native}
Pour simplifier l'utilisation de XGBoost, et uniformiser l'utilisation de cette dernière avec XGBoost, les auteurs des packages ont inclus une fonction de cross-validation directement dans XGBoost.

Cette dernière permet alors de reprendre tous les paramètres disponibles par XGBoost en apprentissage pour la cross-validation et d'avoir ainsi un fonctionnement continu. On retrouve en particulier la possibilité d'utiliser une fonction de perte personnalisée\mysidenote{Plus de détails sur les paramètres de XGBoost seront proposés en Section~\ref{sec:params}, on peut également trouver un descriptif de tous les paramètres en cross-validation à \cite{bib:cv-params}.}.
