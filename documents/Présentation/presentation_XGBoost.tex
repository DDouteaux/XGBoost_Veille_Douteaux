\iffalse
\documentclass{article}

\usepackage{fontspec}
\setmainfont{AR CENA}

\begin{document}
	Bonjour en Arial
\end{document}
\fi






\documentclass{beamer}

\newcommand{\paraTitle}[1]{\textcolor{themeColor}{\textbf{\MakeUppercase #1}}\newline}

\usepackage[no-math]{fontspec}

\defaultfontfeatures{Ligatures=TeX}
\frenchspacing

\setsansfont[Path=./fonts/, Extension=.otf,
Numbers=OldStyle,
BoldFont=FiraSans-Medium,
ItalicFont=FiraSans-LightItalic,
BoldItalicFont=FiraSans-MediumItalic
]{FiraSans-Light}
% Normal font
\setmainfont[Path=./fonts/,	Extension=.otf,
Numbers=OldStyle,
BoldFont=FiraSans-Medium,
ItalicFont=FiraSans-LightItalic,
BoldItalicFont=FiraSans-MediumItalic
]{FiraSans-Light}

\title{MOS x.x -- Veille technologique}
\subtitle{XGBoost, origine et applications}
\author{Damien \textsc{Douteaux}}
\date{Vendredi 3 mars 2017}

\usetheme{texsx}

\begin{document}

\begin{frame}[plain]
	\titlepage
\end{frame}

\section{Sommaire}

\begin{frame}
	\frametitle{Sommaire}
	\tableofcontents
\end{frame}

\section{Qu'est-ce que XGBoost?}
\begin{frame}
	\frametitle{Orgines}
	salut
\end{frame}


\end{document}